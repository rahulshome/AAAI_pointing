\section{Human Evaluation of Instructions}

After designing and conducting our experiments, we became concerned that subjects might regard imprecise referential pointing as understandable but unnatural.  If they did, their judgments might combine ordinary interpretive reasoning with additional effort, self-consciousness or repair.  We therefore added a separate evaluation to assess how natural the generated pointing actions and instructions are. We recruited 480 subjects from Mechanical Turk using the same protocol described in our Data Collection procedure, and asked them to rank how natural they regarded the instruction on a scale of \textit{0 to 5}. 

The examples were randomly sampled from the videos of the referential pointing trials that we showed to subjects for both the Baxter and Kuka robots. These examples were selected in a way that we obtained equal number of samples from each cone. The average rating for samples from the $\ang{45}$, $\ang{67.5}$ and $\ang{90}$ cone are $3.625, 3.521$
and $3.650$ respectively. For Kuka, the average rating for samples from the $\ang{45}$, $\ang{67.5}$ and $\ang{90}$ cone are $3.450, 3.375$, and $3.400$. Overall, the average for Baxter is $3.600$, and for Kuka is $3.408$. The differences between Kuka and Baxter and the differences across cones are not statistically significant ($t \leq |1.07|, p > 0.1 $).  Thus we have no evidence that subjects regard imprecise pointing as problematic.
% Malihe: exact numbers are:
% $t \leq |1.07|, p \geq 0.88 $