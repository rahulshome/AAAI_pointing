%File: formatting-instruction.tex
\documentclass[letterpaper]{article}
\usepackage{aaai}
\usepackage{times}
\usepackage{helvet}
\usepackage{courier}
\usepackage{amsmath}
\usepackage{amsfonts}
\usepackage{graphicx}
\usepackage{xcolor}
\usepackage{float}
\usepackage{siunitx}
\usepackage{multirow}
\newtheorem{hyp}{Hypothesis}
\DeclareMathOperator*{\argmax}{arg\,max}
\DeclareMathOperator*{\argmin}{arg\,min}
\newcommand{\rahul}[1]{{\color{red} \textbf{Rahul:} #1}}
% \newcommand{\rahul}[1]{{\color{red} \textbf{Rahul:} }}
% \newcommand{\mali}[1]{{\color{green} \textbf{Mali:} #1}}
\frenchspacing
\setlength{\pdfpagewidth}{8.5in}
\setlength{\pdfpageheight}{11in}
\pdfinfo{
/Title (Insert Your Title Here)
/Author (Put All Your Authors Here, Separated by Commas)}
\setcounter{secnumdepth}{0}  
 \begin{document}
% The file aaai.sty is the style file for AAAI Press 
% proceedings, working notes, and technical reports.
%
\title{That and There: Judging the Intent of Pointing Actions with Robotic Arms}

% \title{Interpretation of Pointing Action: Expressing the intent of pick-and-place task}

\author{AAAI Press\\
Association for the Advancement of Artificial Intelligence\\
2275 East Bayshore Road, Suite 160\\
Palo Alto, California 94303\\
}
\maketitle
\begin{abstract}
\begin{quote}

Collaborative robotics requires effective communication between a robot and a human partner. Our work identifies limitations of existing models and proposes a set of interpretive principles for how a robotic arm can use pointing actions to communicate task information to people. These principles are evaluated using experiments where English-speaking human subjects view animations of simulated robots instructing pick-and-place tasks. The evaluation distinguishes two classes of pointing actions: referential pointing (identifying objects) and spatial pointing (identifying locations), both of which arise in pick-and-place tasks. Human subjects show greater flexibility in interpreting the intent of referential pointing compared to spatial pointing, which needs to be more deliberate. Our results also demonstrate the effects of variation in the environment and task context on the interpretation of pointing. The corpus and the experiments described in this work have implications for models of context and coordination as well as the effect of commonsense reasoning in human-robot interactions.



%%%%%%%%%%%%%%%
% Referential pointing to an object is easier for a human subject to understand, while spatial pointing may require support of additional queues. 
%%%%%%%%%%%%%%%




% There has been a lot of recent interest in the deployment of robotic manipulators in automation tasks, typically in warehouse settings or service robotics. Motion and task planning research has pushed the boundaries of the class of complex manipulation. However, in scenarios like a warehouse production line, if a robot is unable to service the current request, a human collaborator is typically available to fix the setup in order to resume the automated operation. Such a task the human performs will involve a set of pick-and-place actions in the robot's workspace. The question the current work raises is whether a robot can communicate to a human what the task specification is that the robot needs her to do.
% % (Figure \ref{fig:examples}(e) presents a snapshot of a video that simulates the robot's motion.) 
% The work builds a timed sequence of pointing gestures and motions of the robot end-effector that expresses the sequence of intended pick-and-place tasks to the human observer. 

% \mali{to do after the meeting: 

% - make sure to talk about the significance in Robotics.
% - make sure to mention the ambiguities that can arise when working in a 2D space. I think we need to include the image of the two scenes in the Spatial pointing experiment to show why we see a line.   
% - 


% }


% There has been significant growth in the applications of robotic agents in various domains, including factory automation, and service robotics. Both of these scenarios involve robots and humans to collaborate to various extents.



% The current work focuses on arm-like robots called manipulators, which are designed to perform a rich set of pick-and-place tasks on target objects. 

% Situations may arise where such a robotic system needs to communicate such a task specification to a human. 

% This entails a referent action to point to the object of interest, and another gesture to express the spatial location of the placement. 



% The work contrasts the difference between referential pointing actions, and spatial pointing actions, using a robotic pick-and-place as a demonstrative tool to evaluate the influence of contextual features of the environment on the content and goal of such actions.

% The current work identifies the limitations of the models studied in previous work, when applied to the domain of collaborative robotics, and goes on to propose a set of axioms that were evaluated using experiments with human subjects, to study the rules of effective communication of the intent of pick-and-place tasks. 

% We propose a mathematical model that enables robots to generate pointing configurations that make the goal object as clear as possible — pointing configurations that are legible. We study the implications of legibility on pointing, e.g. that the robot will sometimes need to trade off efficiency for the sake of clarity.

% of investigate a typical way a robotic arm can express such an intent using pointing gestures.   



% The study highlights that different kinds of pointing actions elicit different interpretations from human observers. The work contrasts the difference between referential pointing actions, and spatial pointing actions, using a robotic pick-and-place as a demonstrative tool to evaluate the influence of contextual features of the environment on the content and goal of such actions.

\end{quote}
\end{abstract}

\section{Introduction}
\label{intro}
Recent years have seen a rapid increase of robotic deployment, beyond traditional applications in cordoned-off workcells in factories, into new, more collaborative use-cases. For example, social robotics and service robotics have targeted scenarios like rehabilitation, where a robot operates in close proximity to a human. Where industrial applications envision full autonomy, these collaborative scenarios will always involve interaction between the robotic systems and humans. Thus, these scenarios require effective communication between robots and people.

When the robot's form permits, researchers can design such interactions using principles informed by research on embodied face-to-face human--human communication.  In particular, by realizing \emph{pointing gestures}, an articulated robotic arm with a directional end-effector can exploit a fundamental ingredient of human communication \cite{kita2003pointing}.  Thus far, roboticists have primarily researched pointing gestures that identify objects in simple settings \cite[a.o.]{han2018placing,holladay2014legible,zhao2016experimental}.  Human pointing exhibits richer and more diverse interpretations \cite{kendon:2004}.  This paper develops an empirically-grounded approach to robotic pointing that covers a wider range of physical settings, task contexts and communicative goals.

Our work has two contributions.  First, we create a systematic dataset, involving over 5000 human judgments, where crowd workers describe their interpretation of animations where simulated robots instruct pick-and-place tasks.  Planned comparisons allow us to compare pointing actions that identify objects (referntial pointing) with those that identify locations (spatial pointing), and allow us to quantify the effect of accompanying speech, task constraints and scene complexity, as well as variation in the spatial content the scene.  This new resource documents important differences in the way pointing is interpreted in different cases for the first time.  For example, referential pointing is typically robust to the exactness of the pointing gesture, whereas spatial pointing is much more sensitive and requires more deliberate pointing to ensure a correct interpretation.  The Experiment Design section explains the different conditions we explore, the power analysis for our preregistered protocol and the process of data collection. 

\begin{figure}[t]
    \centering
    \includegraphics[width=0.4\textwidth, trim={0 0.3in 0 0in},clip]{figures/putthatthere2.png}
    \caption{A pick-and-place task requires a \textit{referential} pointing action to the object (orange cube) at the initial position, and a \textit{spatial} pointing action to a final placement position (dotted cube). Such an action by a robot (in red) can also be accompanied by verbal cues like \textit{"Put that there."}}
    \label{fig:pap}
\end{figure}

Our second contribution is a set of interpretive principles, inspired by the literature on vague communication, that summarize our findings about robot pointing.  Our results suggest that pointing selects from a set of candidate interpretations determined by the type of information specified, the possibilities presented by the scene, and the options compatible with the current task.  In particular, we propose that pointing picks out candidates that are significantly closer to the pointing ray than the available alternatives.  Based on our empirical results, we present a mathematical model that formalizes the relevant notions of ``available alternatives'' and ``significantly closer'' and can be used in designing future systems.  The Analysis section explains and justifies our model and underscores how it differs from previous research.

% \begin{hyp}
% In the domain of communication of between a robot and a human observer referential pointing is dependent on the contextual cues and is robust to the exactness of the pointing gesture, whereas the spatial pointing is far more sensitive, and requires more deliberate pointing for ensuring a correct interpretation.
% \end{hyp}






\section{Related work}
\label{related-work}
\section{Related work}
\label{related-work}
Enabling robots to understand and generate instructions to collaboratively solve tasks with humans is an active area of research in natural language processing and human-robot interaction \cite{cha2018survey,butepage2017human}.   This work focuses on research that looks at the role of pointing gestures in communication.

Initial efforts in robotics have looked at making pointing gestures legible, adapting the process of motion planning so that robot movements are correctly understood as being directed toward the location of a particular object in space \cite{holladay2014legible,zhao2016experimental}.  The current work uses motions that are legible in this sense, and goes on to explore how precise the targeting has to be to signal an intended interpretation.

In natural language processing research, it's common to use an expanded pointing cone to describe the possible target objects for a pointing gesture, based on findings about human pointing \cite{kranstedt2003deixis,rieser2004pointing}.  In cluttered scenes, the pointing cone typically includes a region with many candidate referents.  Understanding and generating object references in these situations involves combining pointing with natural language descriptions \cite{han2018placing,kollar2014grounding}.  While we also find that many pointing gestures are ambiguous and can benefit from linguistic supplementation, our results challenge the assumption of a uniform pointing cone. We argue for an alternative, context-sensitive model.

In addition to gestures that identify objects, we also look at pointing gestures that identify points in space.  The closest related work involves navigation tasks, where pointing can be used to discriminate direction (e.g., left vs right) \cite{mei2016listen,tellex2011understanding}.  The spatial information needed for pick-and-place tasks is substantially more precise. Our findings suggest that this precision significantly impacts how pointing is interpreted and how it should be modeled.





\section{Communicating Pick-and-Place}
\label{problem}
We begin with some formal preliminaries to define pick-and-place tasks and characterize the information required to specify them.
 
\noindent\textbf{Manipulator}: A typical robotic system is endowed with the means to physically interact with its immediate vicinity. A conventional class of such robots are called \textit{manipulators}, of which robotic arms are the prime example. 

\noindent\textbf{Workspace}: The manipulator operates in a three-dimensional workspace 
$\mathcal{W}\subseteq \mathbb{R}^3$. The workspace also contains a stable surface of interest defined by a plane $S\subset\mathcal{W}$ along with various objects. To represent 3D coordinates of workspace positions, we use $x\in\mathcal{W}$. 

\noindent\textbf{End-effector}: The \textit{tool-tips} or \textit{end-effectors} are typically geometries that are part of the manipulator that can interact with objects in the environment. These form a manipulator's chief mode of picking and placing objects of interest. These can range from articulated fingers to suction cups. A subset of the workspace that the robot can \textit{reach} with its end-effector is called the reachable workspace. 

\noindent\textbf{Pick-and-place}: Given a target object in its workspace, a \textit{pick-and-place} task requires the object to be picked up from its initial position and orientation, and placed at a final position and orientation. When a manipulator executes this task in its reachable workspace, it uses its end-effector. 
In the rest of this work we will ignore the effect of different orientations of the objects, by considering objects with sufficient symmetry. Due to this simplification, we can consider the pick-and-place task in terms of a transition from an initial position $x_{init}\in\mathcal{W}$ to a final placement position $x_{final}\in\mathcal{W}$.  Thus, a pick-and-place task can be specified with a tuple
$$ PAP = < o, x_{init}, x_{final} >. $$

\noindent\textbf{Pointing Action}: Within its reachable workspace the end-effector of the manipulator can attain different orientations to fully specify a reachable \textit{pose} $p$, which describes its position and orientation.  The robots we study have a directional tooltip that viewers naturally see as projecting a ray $r$ along its axis outward into the scene.  In understanding pointing as communication, the key question is the relationship between the ray $r$ and the spatial values $x_{init}$ and $x_{final}$ that define the pick-and-place task.

To make this concrete, we distinguish between the \emph{target} of pointing and the \emph{intent} of pointing.
Given the ray $r$ coming out of the end-effector geometry, we define the target of the pointing as the intersection of this ray on the stable surface, $$x^*= r\cap S.$$
% Initially we shall describe this relationship of \textit{pointing} loosely, and narrow down on what it entails through our study.
Meanwhile, the intent of pointing specifies one component of a pick-and-place task.  There are two cases.
\begin{itemize}
    \item [-] \textit{Referential Pointing:} The pointing action is intended to identify a target object $o$ to be picked up. This object is the \textit{referent} of such an action. We can find $x_{init}$, based on the present position of $o$.
    \item [-] \textit{Spatial Pointing:} The pointing action is intended to identify the location in the workspace where the object needs to be placed, i.e, $x_{final}$.
\end{itemize}

We study effective ways to express intent for a pick-and-place task. In other words, what is the relationship between a pointing ray $r$ and the location $x_{init}$ or $x_{final}$ that it is intended to identify?  To assess these relationships, we ask human observers to view animations expressing pick-and-place tasks and classify their interpretations.  To understand the factors involved, we investigate a range of experimental conditions.


% \noindent\textbf{Hypothesis}: One way to model the accuracy of the pointing action is to measure the precision of the focused area of a pointing gesture, the so-called pointing cone. We challenge this argument by identifying two main behaviours: referential pointing and spatial pointing.  Our main hypothesis is that the interpretation of pointing actions when the target is part of the space is different from when the target is an object. We expect to run some exploratory studies as well to study how the verbal content and arrangement of objects in the gesture space can influence the interpretation of the pointing action in these two scenarios.



\section{Experiments}
\label{experiments}
This section first describes the commonalities of the experiments that were designed to test our hypothesis, in terms of the experimental setup. The specific variants of the experiments that were run is described as a part of the methods. 
All of the experiments described in this section together with the methods that we have chosen to analyze the data based on a private but approved pre-registration on \textit{aspredicted.org}. We will make the pre-registration public once the anonymity period ends.

\subsection{Experiment Setup}
The setup of the experiment describes the robots used, the environment with the objects and the procedures to generate motions and actions necessary to design different evaluation conditions.

\noindent\textbf{Robotic Platforms}: The experiments were performed on two separate robotic platforms, a \textit{Rethink Baxter}, and a \textit{Kuka IIWA14}.
The \textit{Baxter} is a dual-arm manipulator with two arms mounted on either side of a static torso. The experiments only move the right arm of the \textit{Baxter}. The \textit{Kuka} is consists of a single arm that is vertically mounted, i.e., points upward at the base. In the experiments the robots are fitted with a singly fingered tooltip, where pointing gestures are designed in terms of the outward ray coming out of the end-effector.



\noindent\textbf{Workspace Setup}: Objects are placed in front of the manipulators. In certain trials a table is placed in front of the robot as well, and the objects rest in stable configurations on top of the table. A pick-and-place task is provided specified in terms of the positions of one of the objects. 

\noindent\textbf{Objects}: The objects used in the study include small household items like mugs, saucers and boxes (cuboids), that are all placed in from on the robots.

\noindent\textbf{Motion Generation}: The end-effector of the manipulator is instructed to move to pre-specified waypoints that typically lie between the base of the manipulator and the object itself. Such waypoints fully specify both the position and orientation of the end-effector to satisfy \textit{pointing actions}. The motions are performed by solving \textit{Inverse-Kinematics} for the end-effector geometry and moving the manipulator along these waypoints using a robotic motion planning library \cite{littlefield2014extensible}. The motions were replayed on the model of the robot, and rendered in \textit{Blender}.

% \begin{figure}[t]
%     \centering
%     \includegraphics[width=0.23\textwidth]{baxter.jpg}
%     \includegraphics[width=0.23\textwidth]{kuka.jpg}
%     \caption{The robots used in the study are the \textit{Rethink Baxter}(left) and the \textit{Kuka IIWA14}(right)(<references>)}
%     \label{fig:robots}
% \end{figure}

\begin{figure}[th!]
    \centering
    \includegraphics[width=0.5\textwidth]{pointing_diagram}
    \caption{(A) Workspace setup showing the pointing cone and the corresponding conic section on the table. (B) Shows the degrees-of-freedom considered for placement of the object on the table (C) Sampling policy to sample object poses within the conic section.}
    \label{fig:pointing}
\end{figure}

\begin{figure}[h!]
    \centering
    % \includegraphics[width=0.5\textwidth]{spatial-referential.pdf}
    \includegraphics[height=0.20\textwidth]{figures/kuka_ref.png}
    \includegraphics[height=0.20\textwidth]{figures/kuka_spatial.png}
    \caption{Scenes from the simulated setup showing the differences between referential (\textit{left}) and spacial pointing (\textit{right}), demonstrated on a second robotic manipulator, \textit{Kuka IIWA14}.}
    \label{fig:spatial}
\end{figure}

\begin{figure}[h!]
    \centering
    \includegraphics[width=0.45\textwidth]{natural.pdf}
    \caption{(Left) In an unnatural scene, a gesture pointing to an unstable position (edge of the stack) is deemed correct. (Right) In natural scenes, although the robot points to the edge of the stack, a physically realistic object position gets more user vote than the unstable position.}
    \label{fig:natural}
\end{figure}

% \noindent\textbf{Pointing Action Generation}: Pointing gesture is modeled by a cone $C(r, \theta)$ coming out of the pointing finger where $r$ represents the axis and $\theta$ represents the vertex angle of the cone. As illustrated in Fig~\ref{fig:pointing}, a robot is using it's finger to point to an object placed on a rest surface. For the scope of this study we assume that the rest surface is a plane represented by $S$. Thus the conic section formed on the surface of the table is given by $P=C \cap S$.

% To identify the geometry for the correct pointing gesture in different cases, $N$ object poses $p_i, i=1:N$ are sampled within the conic section $P$. While $p_i$ is the 6d pose for the object with translation $t \in R^3$ and orientation $R \in SO(3)$ only 3 degrees-of-freedom $(x, y, yaw)$ are varied for the experiments. By fixing the $z$ coordinate and the z-axis of rotation, it is ensured that the object rests in a physically stable configuration on the table.

% To sample $N$ poses, the largest eclipse that fits into the conic section $P$ is divided into 4 quadrants $q_1:q_4$ (See Figure~\ref{fig:pointing} (C)) . Within each quadrant $q_i$ the $N/4$ $(x,y)$ positions are sampled uniformly at random. For each placement $yaw$ is randomly sampled from the domain $(0, 2\pi)$. 

\noindent\textbf{Pointing Action Generation}: Pointing gesture is modeled by a cone $C(r, \theta)$ coming out of the pointing finger where $r$ represents the axis and $\theta$ represents the vertex angle of the cone. As illustrated in Fig~\ref{fig:pointing}, a robot is using it's finger to point to an object placed on a rest surface. For the scope of this study we assume that the rest surface is a plane represented by $S$. The part of the conic section that lies on the surface of the table is given by $P=C \cap S$.

To identify the geometry for the correct pointing gesture in different cases, $N$ object poses $p_i, i=1:N$ are sampled within $P$. While $p_i$ is the 6d pose for the object with translation $t \in R^3$ and orientation $R \in SO(3)$ only 2  degrees-of-freedom $(x, y)$ corresponding to $t$ are varied in the experiments. By fixing the $z$ coordinate for translation and restricting the z-axis of rotation to be perpendicular to $S$, it is ensured that the object rests in a physically stable configuration on the table.

The $N$ object poses are sampled by fitting an ellipse within $P$ and dividing the ellipse into 4 quadrants $q_1:q_4$ (See Figure~\ref{fig:pointing} (C)) . Within each quadrant $q_i$ the $N/4$ $(x,y)$ positions are sampled uniformly at random. For certain experiments additional samples are generated with an objective to increase coverage of samples within the ellipse by utilizing a dispersion measure.

% Malihe needs to check this.


\noindent\textbf{Speech}: Some experiments also included verbal cues with phrases like '\textit{Put that there}' along with the pointing actions. It was very important for the pointing actions and these verbal cues to be in synchronization. To fulfill this we generate the voice using Amazon Polly with text written in SSML format and make sure that peak of the gesture (the moment a gesture comes to a stop) is in alignment with the peak of each audio phrase in the verbal cue. During the generation of the video itself we took note of the peak moments of the gestures and then manipulated the duration between peaks of the audio using SSML to match them with gesture peaks after analyzing the audio with an open-source tool- PRAAT \footnote{www.praat.org}.


\subsection{Experiment Design}

Using the setup of the experiments, videos and images from simulation are generated for different benchmarks, and data is collected to study the differences between the pointing gestures, and test our hypothesis. It should be noted that the restrictions of the a simulated 2D interface of videos, and images, that serve as the form of interaction with human subjects can introduce artifacts of the effect of perspective, which we have attempted to control for.  

\subsection{Data Collection}


Participants are asked to report judgments on the interpretation of the pointing action in each class; referential and spatial.  Each participant undertakes two trials from each class. We introduce a dataset of over 5130 responses to robot pointing actions in this study.

Data collection was performed in \textit{Amazon Mechanical Turk}.
All subjects agreed to a consent form and were compensated at an estimated rate of \textit{USD 20} an hour. The subject-pool was restricted to non-colorblind US citizens. Subjects are presented a rendered video of the simulation where the robot performs one referential pointing action, and one spatial pointing action which amounts to it pointing to an object, and then to a final location. During these executions synchronized speech is included in some of the trials to provide verbal cues.

Then on the same page, subjects see the image that shows the result of the pointing action. They are asked whether the result is (a) correct, (b) incorrect, or (c) ambiguous.  



\subsection{Methods}

To test our hypothesis, we studied the interpretation of the two pointing behaviours in different contexts. Assuming our conjecture and a significance level of 0.05, a sample of 28 people in each condition is enough to detect our effect with a 95\% power.

\paragraph{Referential vs Spatial}
In this trial, to reduce the chances of possible ambiguities, we place only one mug is on the table. The \textit{Baxter} robot points its right arm to the mug and then points to its final position, accompanied by a synchronized verbal cue, '\textit{Put that there.}'

% The final image then presents the accurate final position of the mug. Here, we vary the initial position of the mug that is we sample 8 random points from the three cones explained in section yyy. 

We keep the motion identical across all the trials in this method. 
We introduce a variability in the initial position of the mug by sampling $8$ random positions within conic sections subtending $45^{\circ} , 67.5^{\circ}, $ and $90^{\circ}$, on the surface of the table. New videos are generated for each such position of the mug.
This way we can measure how flexible subjects are to the variation of the initial location of the referent object. 

To test the effect for the spatial pointing action, we test similarly sampled positions around the final pointed location, and display these realizations of the mug as the result images to subjects, while the initial position of the mug is kept perfectly situated. 

 A red cube that is in the gesture space of the robot, and is about twice as big as the mug is placed on the other side of the table as a visual guide for the subjects to see how objects can be placed on the table. 

\noindent\textit{Effect of speech}: In order to test the effect of speech on the disparity between the kinds of pointing actions, a set of experiments were designed under the \textit{Referential vs Spatial} method with and without any speech. All subsequent methods will include verbal cues during their action execution. These cues are audible in the video.


\noindent\textit{Reverse Task}: 

One set of experiments are run for the pick-and-place task with the initial and final positions of the object flipped for the reverse task. This trial is designed to be identical the Referential vs Spatial trials, except for the direction. The motions are still executed on the \textit{Baxter's} right arm. 


\noindent\textit{Different Robotic Arm}:
In order to ensure that the results obtained in this study are not dependent on the choice of the robotic platform or its visual appearance, a second robot - a singly armed industrial \textit{Kuka} manipulator - is evaluated in the Referential vs Spatial study (shown in Figure~\ref{fig:spatial}).

\begin{figure}[t]
    % \vspace{-0.1in}
    \centering
    \includegraphics[width=0.3\textwidth]{figures/clutter_trial.png}
    \caption{A cluttered trial consists of collecting the response from a human subject when the position of the referential pointing action lies between two objects.}
    % \vspace{-0.3in}
    \label{fig:cluttered_trial}
\end{figure}
\paragraph{Cluttered Scene}
To study how the presence of other objects would change the behaviour of referential pointing, we examine the interpretation of the pointing actions when there are multiple objects on the tables. We start with a setup where there are 2 cups placed on the table (similar to the setup in Figure~\ref{fig:cluttered_trial}). One is a target cup placed at position $x_{object}$ and a distractor cup at position $x_{distractor}$. With the robot performing an initial pointing action to a position $x_{init}$ on the table. Both the objects are sampled around $x_{init}$ along the diametric line of the conic section arising from increasing cone angles of $45^\circ, 67.5^\circ, $ and $90^\circ$, where the separation of $x_{object}$, and $x_{distractor}$ is equal to the length of the diameter of the conic section, $D$. The objects are then positioned on the diametric line with a random offset between $[-\frac{D}{2}, \frac{D}{2}]$ around $x_{init}$ and along the line. This means that the objects are at various distances apart, and depending upon the offset, one of the objects is nearer to the pointing action. The setup induces that the nearer cup serves as the \textit{object}, and the farther one serves as the \textit{distractor}. The motions are performed on the \textit{Baxter's} right arm. The camera perspective in simulation is set to be facing into the pointing direction. The subjects in this trial are shown images of the instant of the referential pointing action.




\paragraph{Natural vs Unnatural scene}
In this experiment we study how the contextual and physical understanding of the world impacts the interpretation of pointing gestures. We generate a scenario for spatial pointing in which the right arm of the \textit{Baxter} points to a final placement position for the cuboidal object on top of a stack of cuboidal objects but towards the edge which makes it physically unstable. The final configurations of the object (Figure~\ref{fig:topedgetable}) shown to the users were a) object lying on top of the stack b) object in the unstable configuration towards the edge of the stack and c) object at the bottom of the stack towards one side. New videos are generated for each scenario along with verbal cues.

The pointing action, as well as the objects of interest stay the identical between the natural, and unnatural trials. The difference lies in other objects in the scene that could defy gravity and float in the unnatural trials. The subjects were given a text-based instruction at the beginning of an unnatural trial saying they were seeing a scene where "gravity does not exist". 


% \begin{figure*}[t]
%     \centering
%     \includegraphics[width=0.32\textwidth ] {figures/baxter_Referential_Correct.png}
%     \includegraphics[width=0.32\textwidth ]{figures/baxter_Referential_Incorrect.png}
%     \includegraphics[width=0.32\textwidth ]{figures/baxter_Referential_Ambiguous.png}
%     \includegraphics[width=0.32\textwidth ]{figures/baxter_Spatial_Correct.png}
%     \includegraphics[width=0.32\textwidth ]{figures/baxter_Spatial_Incorrect.png}
%     \includegraphics[width=0.32\textwidth ]{figures/baxter_Spatial_Ambiguous.png}
%     \caption{The results from the referential versus spatial trials for the \textit{Baxter} robot. The locations of the responses correspond to the center of the circles, and are plotted in the coordinate frame centered at the position of the true pointing action, marked with $\times$. The size of the circles are proportional to the number of the responses.}
%     \label{fig:baxtersimple}
% \end{figure*}


% \begin{figure*}[t]
%     \centering
%     \includegraphics[width=0.32\textwidth ] {figures/kuka_Referential_Correct.png}
%     \includegraphics[width=0.32\textwidth ]{figures/kuka_Referential_Incorrect.png}
%     \includegraphics[width=0.32\textwidth ]{figures/kuka_Referential_Ambiguous.png}
%     \includegraphics[width=0.32\textwidth ]{figures/kuka_Spatial_Correct.png}
%     \includegraphics[width=0.32\textwidth ]{figures/kuka_Spatial_Incorrect.png}
%     \includegraphics[width=0.32\textwidth ]{figures/kuka_Spatial_Ambiguous.png}
%     \caption{The results from the referential versus spatial trials for the \textit{Kuka} robot. The locations of the responses correspond to the center of the circles, and are plotted in the coordinate frame centered at the position of the true pointing action, marked with $\times$. The size of the circles are proportional to the number of the responses.}
%     \label{fig:kukasimple}
% \end{figure*}



% \begin{figure*}[t]
%     \centering
%     \includegraphics[width=0.4\textwidth ] {figures/baxter_Referential__color.png}
%     \includegraphics[width=0.4\textwidth ]{figures/baxter_Spatial__color.png}
%     \includegraphics[width=0.4\textwidth ]{figures/kuka_Referential__color.png}
%     \includegraphics[width=0.4\textwidth ]{figures/kuka_Spatial__color.png}
%     \caption{The aggregated results from the referential versus spatial trials for the \textit{Baxter} and \textit{Kuka} robots. Each cell indicates the average responses over a $3\times3$ grid centered on it. The color represents the average fraction of correct responses as green, incorrect as red, and ambiguous ones as blue.}
%     \label{fig:colorsimple}
% \end{figure*}

\paragraph{Different verbs}  
To test if the effect is specific to the verb put, we designed a control experiment where everything remained the same as the Referential vs Spatial trials except the verb \textit{put} which we replaced with \textit{place, move} and \textit{push}. Here again 30 data points for each sampled $x^*$



\section{Analysis}
\label{analysis}
\section{Analysis}
\label{analysis}

\begin{figure*}[ht!]

    \centering
    \includegraphics[width=0.5\textwidth, trim={0 0 0 3.3in},clip ] {figures/labels.png}\\
    \includegraphics[width=0.325\textwidth ] {figures/baxter_Referential_.png}
    \includegraphics[width=0.325\textwidth ] {figures/baxter_Referential-Reverse_.png}
    \includegraphics[width=0.325\textwidth ]{figures/kuka_Referential_.png}
    \includegraphics[width=0.325\textwidth ]{figures/baxter_Locating_.png}
    \includegraphics[width=0.325\textwidth ]{figures/baxter_Locating-Reverse_.png}
    \includegraphics[width=0.325\textwidth ]{figures/kuka_Locating_.png}
    \caption{The aggregated results from the referential versus spatial trials for the \textit{Baxter} and \textit{Kuka} robots. The locations of the responses correspond to the center of the circles, and are plotted in the coordinate frame centered at the position of the pointing action, marked with $\times$. The circles show the fraction of correct (grey), incorrect (black) and ambiguous (white) responses.}
    \label{fig:aggregatesimple}
\end{figure*}

\paragraph{Referential vs Locating}
We study how varying the target of the pointing action from a referent object to a part of the space changes the interpretation of the pointing action by comparing the interpretation of the position of the pointing action $x^*$ in each condition. 

Figure~\ref{fig:aggregatesimple} shows the results of the experiment. The plot shows the spread of \textit{correct, incorrect, ambiguous} responses over the sampled positions about the location of referential vs locating pointing actions. The referential data demonstrates the robustness of the interpretation. Most of the responses were overwhelmingly \textit{correct}, for both robots, in interpreting a referent object in the \textit{pick} part of a pick-and-place task. The locating pointing shows a much higher sensitivity to an accuracy of $x^*$ with respect to the true final placement. This comes up as a larger incidence of \textit{incorrect} and \textit{ambiguous} responses from the human subjects. This trend is true for the reverse trial as well.

While the study attempts to separate out and measure the critical aspects of the interpretation of robotic pointing actions some ambiguities like those arising out of perspective of the camera being projected onto a simulated 2D video or image are unavoidable. We suspect that the observed stretch of \textit{correct} responses in spatial trials is due to perspective.

To test our hypothesis that Referential pointing is interpreted less precisely than Locating pointing we performed a Chi-squared test and compared the proportion of \textit{correct}, \textit{incorrect} and \textit{ambiguous} responses in referential and spatial trials. The results of the test shows that these two classes are statistically significantly different ($\chi^2= 13.89, p = 0.00096$).

To study if we are observing the same effects in the results of the reverse trial, no speech trial and the Kuka trial, we ran an equivalence test following the two one-sided tests method as described in \cite{lakens2017equivalence}, where each test is a pooled $z$-test with no continuity
correction with a significance level of 0.05. We found that changing the robot, removing the speech and changing the direction of the pointing action does not make a difference in the interpretation of locating pointing and referential pointing within any margin that is less than 5\%.


\begin{table}[h]
\label{tab:naturaltrial}
\begin{tabular}{lllll}
          &               & correct     & incorrect & ambiguous \\ \hline
unnatural & top           & 12          & 9         & 9         \\
          & \textbf{edge} & \textbf{24} & 2         & 4         \\
          & table         & 2           & 2         & 26        \\ \hline
natural   & \textbf{top}  & \textbf{26} & 3         & 1         \\
          & edge          & 9           & 11        & 10        \\
          & table         & 7           & 13        & 12        \\ \hline
\end{tabular}
\caption{Results of the unnatural scene and natural scene. (numbers are out of 30.)}
\label{tab:natural-unnatural}
\end{table}

\paragraph{Natural vs Unnatural}

As shown in Table~\ref{tab:natural-unnatural} we observed in the natural scene, when the end-effector points towards the edge of the cube that is on top of the stack, subjects place the new cube on top of the stack or on the table instead of the edge of the cube. However, in the unnatural scene, when we explain to subjects that there is no gravity, a majority agree with the final image that has the cube on the \textit{edge}. To test if this difference is statistically significant, we use the Fisher exact test \cite{10.2307/2340521}. The test statistic value is $0.0478$. The result is significant at $p < 0.05$. 


\paragraph{Different verbs}
The results of the Chi-squared test shows that in spatial trials when we replace \textit{put} with \textit{place}, \textit{push} and \textit{move}, the differences of the distributions of \textit{correct}, \textit{incorrect} and \textit{ambiguous} responses are not statistically significant ($\chi=0.2344 $, $p = 0.971$). The coefficients of the multinomial logistic regression model and the $p$-values also suggest that the differences in judgements 
with different verbs are not statically significant ($b<0.0001$ , $p>0.98$).


\begin{figure}[ht]
    \centering
    \includegraphics[width=\linewidth]{figures/baxter_Clutter_granular.png}
    \caption{The scatter plot represents the spread of responses where human subjects chose the \textit{nearer cup} (green), \textit{farther} cup (red), and ambiguous (white). The \textit{x-axis} represents the absolute difference between the distances of each cup to the locations of pointing, the \textit{y-axis} represents the total distance between the two cups.}
    \label{fig:cluttered}
\end{figure}
\paragraph{Cluttered}
The data from these trials show how human subjects select between the two candidate target objects on the table. Since the instructions do not serve to disambiguate the target mug, the collected data show what the observers deemed as the \textit{correct} target.  Figure~\ref{fig:cluttered} visualizes subjects' responses across trials.  The location of each pie uses the $x$-axis to show how much closer one candidate object is to the pointing target than the other, and uses the $y$-axis to show the overall imprecision of pointing.  Each pie in Figure~\ref{fig:cluttered} shows the fraction of responses across trials that recorded the nearer (green) mug as correct compared to the farther mug (red). The white shaded fractions of the pies show the fraction of responses where subjects found the gesture ambiguous.

As we can see in Figure~\ref{fig:cluttered}, once the two objects are roughly equidistant the cups from the center of pointing (within about 10cm), subjects tend to regard the pointing gesture as ambiguous, but as this distance increases, subjects are increasingly likely to prefer the closer target.  In all cases, wherever subjects have a preference for one object over the other, they subjects picked the mug that was the nearer target of the pointing action more often than the further one.




\section{Human Evaluation of Instructions}

To evaluate how natural are the generated pointing actions and instructions, we recruited 480 subjects from mechanical Turk using the same protocol described in our Data Collection procedure, and asked them to rank how natural they regarded the instruction on a scale of \textit{0 to 5}. 
% asked subject the following question: 

% ``on a scale of 0 to 5, how natural is the instruction?''

The examples were randomly sampled from the videos of the referential pointing trials that we showed to subjects for both the Baxter and Kuka robots. These examples were selected in a way that we obtained equal number of samples from each cone. The average rating for samples from the $\ang{45}$, $\ang{67.5}$ and $\ang{90}$ cone are $3.625, 3.521$
and $3.650$ respectively. For Kuka, the average rating for samples from the $\ang{45}$, $\ang{67.5}$ and $\ang{90}$ cone are $3.450, 3.375$, and $3.400$. Overall, the average for Baxter is $3.600$, and for Kuka is $3.408$. The differences both between Kuka and Baxter, and different cones are not statistically significant ($t \leq |1.07|, p > 0.1 $).
% Malihe: exact numbers are:
% $t \leq |1.07|, p \geq 0.88 $

\section{Design Principles}

The results of the experiments suggest that spatial pointing is interpreted rather precisely, where referential pointing is interpreted relatively flexibly.  This naturally aligns with the possibility for alternative interpretations.  For spatial reference, any location is a potential target.  However, for referential pointing, it suffices to distinguish the target object from its distractors.

We can characterize this interpretive process in formal terms by drawing on observations from the literature on vagueness \cite{kyburg2000fitting,graff2000shifting}.  Any pointing gesture starts from a set of candidate interpretations $D \subset \mathcal{W}$ determined by the context and the communicative goal.  In unconstrained situations, spatial pointing allows a full set of candidates $D = \mathcal{W}.$  If factors like common-sense physics impose task constraints, that translates to restrictions on feasible targets $CS$, leading to a more restricted set of candidates $D = CS \cap \mathcal{W}$.  Finally, for referential pointing, the potential targets are located at locations $x_1 \ldots x_N \in S$, and $D = \{ x_1 \ldots x_N \}.$

Based on the communicative setting, we know that the pointing gesture, like any vague referring expression, must select at least one of the possible interpretations \cite{kyburg2000fitting}.  We can find the best interpretation by its distance to the target $x^*$ of the pointing gesture.  Using $d(x,x^*)$ to denote this distance, gives us a threshold $$\theta = \min_{x \in D} d(x, x^*).$$

Vague descriptions can't be sensitive to fine distinctions \cite{graff2000shifting}.  So if a referent at $\theta$ is close enough to the pointing target, then another at $\theta + \epsilon$ must be close enough as well, for any value of $\epsilon$ that is not significant in the conversational context.  Our results suggest that viewers regard 10cm (in the scale of the model simulation) as an approximate threshold for a significant difference in our experiments.

In all, we predict that a pointing gesture is interpreted as referring to $\{x \in D | d(x,x^*) \leq \theta + \epsilon\}.$  We explain the different interpretations through the different choice of $D$.

\paragraph{Spatial Pointing.}  For unconstrained spatial pointing, $x^* \in D$, so $\theta=0$.  That means, the intended placement cannot differ significantly from the pointing target.  Taking into account common sense, we allow for small divergences that connect the pointing, for example, to the closest stable placement.

\paragraph{Referential Pointing.}  For referential pointing, candidates play a much stronger role.  A pointing gesture always has the closest object to the pointing target as a possible referent.  However, ambiguities arise when the geometries of more than one object intersect with the $\theta+\epsilon$-neighborhood of $x^*$.   We can think of that, intuitively, in terms of the effects of $\theta$ and $\epsilon$.  Alternative referents give rise to ambiguity not only when they are too close to the target location ($\theta$) but even when they are simply not significantly further away from the target location ($\epsilon$).  

\section{Conclusion}
\label{conclusion}

We have presented an empirical study of the interpretation of simulated robots instructing pick-and-place tasks.  Our results show that robots can effectively combine pointing gestures and spoken instructions to communicate both object information and spatial information---and offer the first empirical characterization of the use of robot gestures to communicate precise spatial locations.  The dataset and a demo of the experiments are attached with this submission, and will be released upon acceptance of the paper.

We have suggested that pointing, like other vague references, select the candidates that are 
It should be noted that the restrictions of the a simulated 2D interface of videos, and images, that serve as the form of interaction with human subjects can introduce artifacts of the effect of perspective.  








\bibliography{pointing} 
\bibliographystyle{aaai}



\end{document}